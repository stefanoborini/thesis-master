\section{Approssimazione Hartree-Fock e metodo SCF}

Nelle sezioni precedenti si \`e fatto riferimento alla possibilit\`a di
ottenere la migliore descrizione possibile per lo stato fondamentale di
un sistema sfruttando il teorema variazionale.
Questa possibilit\`a passa attraverso l'ottimizzazione
di certi parametri caratterizzanti la funzione d'onda stessa.
L'approccio visto nella sezione \ref{sec:antisimmetria} ha permesso inoltre
di delineare una possibile ed efficiente rappresentazione di una funzione
d'onda polielettronica come combinazione lineare di entit\`a antisimmetrizzate,
i determinanti di Slater, ciascuno dei quali rappresenta una particolare
disposizione elettronica sugli spinorbitali monoelettronici deputati a base del
nostro sistema.

Una prima e semplicistica approssimazione, che tuttavia fornisce buoni risultati
per sistemi semplici ed in determinate condizioni, \`e considerare l'espansione
\beqa
\label{eqn:espansione}
\Psi(1,2,\ldots,n) &=& \sumidx{i_1 < i_2 < \ldots < i_n } c_{i_1,i_2,\ldots,i_n} \detsl{\psi_{{i}_1} \psi_{{i}_2} \ldots \psi_{{i}_n}} \\
&=& \sumidx{I} c_I \Phi_I
\eeqa
limitata ad un solo determinante, ovvero si ammette
\beqas
c_I &=& 0  \qquad \forall I \neq 0 \\
c_0 &=& 1
\eeqas
e di conseguenza
\beq
\Psi(1,2,\ldots,n) = \Phi_0
\eeq
Questa drastica approssimazione funziona piuttosto bene in molecole
nello stato di singoletto, in cui gli orbitali spaziali sono doppiamente
occupati, ed in prossimit\`a della geometria di equilibrio.
Il processo che porta all'individuazione dei migliori orbitali molecolari
va sotto il nome di \textit{Self Consistent Field} od ottimizzazione SCF.
La funzione d'onda polielettronica
\beqa
\Psi(1,2,\ldots,n) &=& \detsl{\psi_1 \psi_2 \ldots \psi_n} \nonumber \\
&=& (n!)^{-1/2} \mbox{det}\left| \psi_1 \psi_2 \ldots \psi_n \right|
\eeqa
dovr\`a quindi essere ottimizzata sui singoli spinorbitali $\psi_i$.

\`E dimostrabile che i migliori spinorbitali per l'approssimazione
monodeterminantale sono le autofunzioni di un operatore monoelettronico
$\fock$ detto operatore di Fock. L'equazione
\beq
\label{eqn:fock}
\fock \psi = \epsilon \psi
\eeq
\`e detta \textit{equazione di Hartree-Fock} e fornisce un set di
spinorbitali monoelettronici che meglio descrivono la situazione elettronica
rappresentata da un singolo determinante. Tali spinorbitali sono detti
\textit{canonici}.

L'autovalore corrispondente $\epsilon$ ha una diretta interpretazione fisica,
in quanto rappresenta, cambiato di segno, un'approssimazione dell'energia
necessaria a rimuovere l'elettrone presente nello spinorbitale $\psi$,
detta energia di ionizzazione per l'estrazione dell'elettrone dall'orbitale $\psi$.

L'equazione \ref{eqn:fock} possiede, su uno spazio spinorbitalico infinito,
infinite soluzioni. Di queste, solo un numero pari agli elettroni
presenti, e precisamente quelle a pi\`u bassa energia orbitalica $\epsilon$,
sono occupate. Il restante spazio complementare \`e detto \textit{spazio
virtuale}, a cui appartengono spinorbitali vuoti.
In un caso reale, il numero di orbitali \`e limitato dalla
dimensionalit\`a della base scelta. In generale, tale base deriva da un set
di $k$ orbitali spaziali $\{ \phi_i \}$ che generano $ 2k $ spinorbitali $\{
\phi_i \alpha , \phi_i \beta \}$. Per $ k \rightarrow \infty $, l'algoritmo
SCF possiede via via maggiori gradi di libert\`a su cui effettuare
l'ottimizzazione HF, e di conseguenza l'energia attesa $ E_0 =
\braket{\Psi_0}{\ham}{\Psi_0} $ tende a diminuire fino a raggiungere un limite
denominato \textit{limite di Hartree-Fock}.
Come conseguenza, per qualsiasi $k$ intero positivo ragionevole, l'energia
sar\`a sempre maggiore del limite di Hartree-Fock.

\subsection{Teorema di Brillouin}

L'equazione Hartree-Fock fornisce un set di spinorbitali, di cui $n$
sono utilizzati per costruire $\ket{\Psi_0}$. Questa funzione d'onda non \`e tuttavia l'unica
disposizione elettronica attuabile con il set dato, come abbiamo visto
nell'equazione \ref{eqn:espansione}. Ogni ulteriore disposizione
elettronica attuabile pu\`o essere pensata come eccitazione di un dato 
numero di elettroni dagli orbitali occupati del determinante HF agli
orbitali virtuali vuoti.
Con tale premessa, l'espansione \ref{eqn:espansione} pu\`o essere
formulata come
\beqa
\label{eqn:mr}
\Psi &=& c_0 \ket{\Phi_0} + \sumidx{ra} c_a^r \Phi_a^r + \sumidx{rsab} c_{ab}^{rs} \Phi_{ab}^{rs} + \ldots
\eeqa
dove gli indici $r$ ed $a$ intendono la promozione di un elettrone
dallo spinorbitale $\psi_a$ allo spinorbitale $\psi_r$.
Volendo descrivere la funzione d'onda come combinazione di determinanti, 
nel tentativo di migliorarne la descrizione, si potrebbe troncare la somma
\ref{eqn:mr} ai contributi singolarmente eccitati diagonalizzando
l'hamiltoniano nello spazio del determinante di Fock e delle sue singole
eccitazioni $\{ \Phi_0 , \{ \Phi_a^r \} \}$:
\beq
\left(
\begin{array}{cc}
\braket{\Phi_0}{\ham}{\Phi_0} & \braket{\Phi_0}{\ham}{\Phi^r_a} \\
& \\
\braket{\Phi^r_a}{\ham}{\Phi_0} & \braket{\Phi^r_a}{\ham}{\Phi^{r^{\prime}}_{a^{\prime}}} \\
\end{array}
\right)
\left(
\begin{array}{c}
c_0 \\
c^r_a \\
\end{array}
\right) = E_0
\left(
\begin{array}{c}
c_0 \\
c^r_a \\
\end{array}
\right)
\eeq
Il teorema di Brillouin (Cfr. \cite{asi-71-1933} e \cite{asi-159-1934})
assicura che, se gli orbitali soddisfano l'equazione di Hartree-Fock, tutte
le interazioni $\braket{\Phi_0}{\ham}{\Phi^r_a}$ sono nulle, e dunque
l'energia non \`e migliorabile ulteriormente limitando l'espansione
\ref{eqn:mr} alle sole singole eccitazioni.

