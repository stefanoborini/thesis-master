\section{Teoria della perturbazione}
\label{sec:perturbazione}

Sia dato l'hamiltoniano vero $\ham$. \`E possibile esprimere tale hamiltoniano come
una espansione di Taylor interrotta al primo termine
\beq
\label{eqn:pert1}
\ham = \ham^{(0)} + \lambda \hat{V}
\eeq
dove $\ham^{(0)}$ \`e un hamiltoniano modello, $\lambda$ \`e un parametro numerico che
esprime l'intensit\`a dell'effetto perturbativo e $\hat{V}$ \`e un
operatore perturbativo.

Nell'ipotesi di conoscere tutti gli autovalori e autovettori dell'hamiltoniano modello
$\ham^{(0)}$, ovvero di conoscerne la sua decomposizione spettrale, si avr\`a
\beq
\label{eqn:pert2}
\ham^{(0)} \Psi_n^{(0)} = E_n^{(0)} \Psi_n^{(0)} \quad n=0,1,2,\ldots
\eeq
dove le $\Psi_n^{(0)}$ (abbreviate in $\ket{n}$) costituiscono una base
completa di funzioni d'onda.

Ovviamente, in assenza di azione perturbativa la funzione descrittiva per lo stato
$n$ sar\`a $\Psi_n^{(0)}$ e la sua energia $E_n^{(0)}$, come indicato
dall'equazione \ref{eqn:pert2}. In seguito alla perturbazione, la descrizione
dello stato e dell'energia cambieranno, e saranno funzioni del parametro
perturbativo $\lambda$: per la funzione d'onda sar\`a
\beq
\label{eqn:pert2-1}
\Psi_n = \Psi_n^{(0)} + \lambda \Psi_n^{(1)} + \lambda^2 \Psi_n^{(2)} + \ldots
\eeq
e per l'energia
\beq
\label{eqn:pert2-2}
E_n = E_n^{(0)} + \lambda E_n^{(1)} + \lambda^2 E_n^{(2)} + \ldots
\eeq
ovvero espansioni di Taylor sul parametro $\lambda$.

Dal momento che l'equazione di cui si ricerca soluzione \`e
\beq
\ham \Psi_n = E_n \Psi_n
\eeq
attraverso opportuna sostituzione, utilizzando le relazioni \ref{eqn:pert2-1}
e \ref{eqn:pert2-2} e successivamente raccogliendo le potenze di $\lambda^n$
si ottiene
\beqa
& & \lambda^0 \left( \ham^{(0)} \Psi_n^{(0)} - E_n^{(0)} \Psi_n^{(0)} \right) \nonumber \\ 
&+& \lambda^1 \left( \ham^{(0)} \Psi_n^{(1)} + \hat{V} \Psi_n^{(0)} - E_n^{(0)} \Psi_n^{(1)} - E_n^{(1)} \Psi_n^{(0)} \right) \nonumber \\
&+& \lambda^2 \left( \ham^{(0)}\Psi_n^{(2)} + \hat{V}\Psi_n^{(1)} - E_n^{(0)} \Psi_n^{(2)} - E_n^{(1)} \Psi_n^{(1)} - E_n^{(2)} \Psi_n^{(0)} \right) \nonumber \\
&+& \ldots = 0
\eeqa
Affinch\'e tale uguaglianza sia soddisfatta, \`e necessario che ogni coefficiente di $\lambda$ sia nullo.
Ne risultano quindi le seguenti equazioni
\beqa
\label{eqn:pert3}
\ham^{(0)} \Psi_n^{(0)} &=& E_n^{(0)} \Psi_n^{(0)} \\
\label{eqn:pert4}
\left( \ham^{(0)} - E_n^{(0)} \right) \Psi_n^{(1)} &=& \left( E_n^{(1)} - \hat{V} \right) \Psi_n^{(0)} \\
\label{eqn:pert5}
\left( \ham^{(0)} - E_n^{(0)} \right) \Psi_n^{(2)} &=& \! E_n^{(2)} \Psi_n^{(0)} \! + \left( E_n^{(1)} - \hat{V} \right) \Psi_n^{(1)} \\
& \vdots & \nonumber
\eeqa
La soluzione dell'equazione \ref{eqn:pert3} \`e conosciuta e fornisce l'ordine zero della perturbazione.
L'equazione \ref{eqn:pert4} rende invece conto della correzione perturbativa al prim'ordine della
funzione. 
\`E possibile esprimere la correzione al primo ordine $\Psi_n^{(1)}$ come 
combinazione lineare di funzioni all'ordine zero, in quanto tale set
\`e una base per lo spazio funzionale trattato. Di conseguenza
\beq
\Psi_n^{(1)} = \sumidx{k}c_k \Psi_k^{(0)}
\eeq
Sostituendo questa espressione nell'equazione \ref{eqn:pert4} si ottiene
\beqa
\sumidx{k} c_k \left( \ham^{(0)} - E_n^{(0)} \right) \ket{k} &=& \left( E_n^{(1)} - \hat{V} \right) \ket{n} \nonumber \\
\label{eqn:pert6}
\sumidx{k} c_k \left( E_k^{(0)} - E_n^{(0)} \right) \ket{k} &=& \left( E_n^{(1)} - \hat{V} \right) \ket{n}
\eeqa
ed applicando ora il bra $\bra{n}$
\beqa
\sumidx{k} c_k \left( E_k^{(0)} - E_n^{(0)} \right) \delta_{nk} &=& E_n^{(1)} - \braket{n}{\hat{V}}{n} \nonumber \\
0 &=& E_n^{(1)} - \braket{n}{\hat{V}}{n} \nonumber \\
E_n^{(1)} &=& \braket{n}{\hat{V}}{n}
\eeqa
si ottiene la correzione dell'energia al primo ordine.

Per ottenere la correzione della funzione d'onda al primo ordine, \`e sufficiente applicare il bra $\bra{l}$,
con $l$ generico diverso da $\bra{n}$, all'equazione \ref{eqn:pert6}
\beqa
\sumidx{k} c_k \left( E_k^{(0)} - E_n^{(0)} \right) \delta_{kl} &=& E_n^{(1)} \integral{l}{n} - \braket{l}{\hat{V}}{n} \nonumber \\
c_l \left( E_l^{(0)} - E_n^{(0)} \right) &=& - \braket{l}{\hat{V}}{n} \nonumber \\
\label{eqn:pert7}
c_l &=& \frac{\braket{l}{\hat{V}}{n}}{E_n^{(0)} - E_l^{(0)}}
\eeqa
L'espressione \ref{eqn:pert7} \`e limitata a casi non degeneri (nel qual caso $E_n^{(0)} - E_l^{(0)}$ potrebbe essere $0$ per $l \neq n$).

Si otterr\`a una migliore rappresentazione della funzione $\Psi_n$
\beq
\Psi_n = \Psi_n^{(0)} + \sumidx{k \neq n} \left( \frac{\braket{k}{\hat{V}}{n}}{E_n^{(0)} - E_k^{(0)}} \right) \Psi_k^{(0)}
\eeq
L'effetto perturbativo di conseguenza introduce, per meglio descrivere la funzione
d'onda, dei livelli eccitati appartenenti all'ordine zero secondo degli
opportuni coefficienti.

