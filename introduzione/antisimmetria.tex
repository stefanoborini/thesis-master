% DONE
\section{Requisito di antisimmetria}
\label{sec:antisimmetria}

Una legge naturale generale indica che, per particelle fermioniche quali gli elettroni,
la funzione d'onda descrittiva di uno stato deve necessariamente essere antisimmetrica
in seguito allo scambio di due particelle. Pi\`u in dettaglio, nel caso della semplice
funzione bielettronica $\Psi(1,2)$, deve valere
\beq
\label{eqn:antisimm1}
\Psi(2,1) = - \Psi(1,2)
\eeq
ovvero lo scambio di due particelle si traduce in una variazione del segno
della funzione descrittiva dello stato.

Data una funzione polielettronica, \`e sempre possibile esprimere tale funzione
come opportuna combinazione di prodotti di funzioni monoparticella. Questa possibilit\`a deve
tuttavia soddisfare il vincolo naturale di antisimmetria.
Al fine di garantire ci\`o, sia $\Psi(1,2,\ldots,n)$ una funzione a $n$ particelle.
Parametrizzando la posizione delle particelle $(2,\ldots,n)$, la funzione
ottenuta, presentando una sola variabile dinamica, pu\`o essere espansa su una base
spinorbitalica monoparticella $\left\{ \psi_i \right\} $ ortonormale
\beq
\Psi(1,\underbrace{2,\ldots,n}_{\mbox{fissate}}) = \Psi(1) = \sumidx{i} c_i(2,\ldots,n) \psi_i(1)
\eeq
Nei coefficienti $c_i$ risiede la dipendenza dalle variabili fissate, e quindi tali coefficienti
sono funzioni dei parametri. \`E possibile ripetere la procedura sulla
funzione $c_i(2,\ldots,n)$, parametrizzando le coordinate $(3,\ldots,n)$ ed espandendola
sulla medesima base spinorbitalica
\beq
c_i(2,\underbrace{3,\ldots,n}_{\mbox{fissate}}) = c_i(2) = \sumidx{j} c^{\prime}_{ij}(3,\ldots,n) \psi_j(2)
\eeq
Ripetendo la procedura per le $n$ coordinate, si ottiene
\beq
\Psi(1,2,\ldots,n) = \sumidx{i,j,k,\ldots,l} c_{ijk\ldots l} \psii(1) \psij(2) \psik(3) \ldots \psi_l(n)
\eeq
con $c_{ijk\ldots}$ coefficiente puramente numerico.
Applicando quanto detto ad un caso a due particelle si ha
\beqa
\Psi(1,2) &=& \sumidx{i,j} c_{ij} \psii(1) \psij(2) \nonumber \\
          &=& c_{11} \psi_1(1) \psi_1(2) + c_{12} \psi_1(1) \psi_2(2) + c_{21} \psi_2(1) \psi_1(2) \nonumber \\
	  &&  + c_{22} \psi_2(1) \psi_2(2) + \ldots
\eeqa
La necessit\`a di soddisfare l'equazione \ref{eqn:antisimm1} impone di
considerare anche
\beqa
\Psi(2,1) &=& c_{11} \psi_1(2) \psi_1(1) + c_{12} \psi_1(2) \psi_2(1) + c_{21} \psi_2(2) \psi_1(1) \nonumber \\
	  &&  + c_{22} \psi_2(2) \psi_2(1) + \ldots
\eeqa
e di conseguenza
\beqa
c_{ii} &=& - c_{ii} \rightarrow c_{ii} = 0 \nonumber \\
c_{ij} &=& - c_{ji} \nonumber
\eeqa
Quindi, la sommatoria si riduce a
\beqa
\Psi(1,2) &=& c_{12} \left( \psi_1(1) \psi_2(2) - \psi_2(1) \psi_1(2) \right) \nonumber \\
	  && + c_{13} \left( \psi_1(1) \psi_3(2) - \psi_3(1) \psi_1(2) \right) + \ldots \nonumber \\
	  &=& \sumidx{i<j} c_{ij} \left( \psi_i(1) \psi_j(2) - \psi_j(1) \psi_i(2) \right) \nonumber \\
	  &\rightarrow& \sumidx{i<j} c_{ij} \detsl{\psii \psij}
\eeqa
dove, con la scrittura
\beq
\detsl{\psii \psij} = 2^{-\frac{1}{2}} \left|
\begin{array}{cc}
\psii(1) & \psij(1) \\
\psii(2) & \psij(2) \\
\end{array}
\right|
\eeq
intendiamo, previa normalizzazione, il \textbf{determinante di Slater} normalizzato per un sistema bielettronico.
Generalizzando a $n$ elettroni, si avr\`a
\beqa
\Psi(1,2,\ldots,n) &=& \sumidx{i_1 < i_2 < \ldots < i_n } c_{i_1,i_2,\ldots,i_n} \detsl{\psi_{{i}_1} \psi_{{i}_2} \ldots \psi_{{i}_n}} \\
&=& \sumidx{I} c_I \Phi_I
\eeqa
con
\beqa
\detsl{\psi_{{i}_1} \psi_{{i}_2} \ldots \psi_{{i}_n}} &=&
(n!)^{-\frac{1}{2}} \left|
\begin{array}{cccc}
\psi_{{i}_1}(1) & \psi_{{i}_2}(1) & \ldots & \psi_{{i}_n}(1) \\
\psi_{{i}_1}(2) & \psi_{{i}_2}(2) & \ldots & \psi_{{i}_n}(2) \\
\vdots          &  \vdots         & \ddots &  \vdots          \\
\psi_{{i}_1}(n) & \psi_{{i}_2}(n) & \ldots & \psi_{{i}_n}(n) \\
\end{array}
\right|
\eeqa

Concludendo, una funzione d'onda pu\`o essere espressa come una combinazione
lineare di determinanti di Slater, ciascuno dei quali descrive una possibile
scelta di $n$ spinorbitali da occupare (configurazione elettronica).
Il determinante di Slater soddisfa automaticamente il requisito di
antisimmetria, dal momento che lo scambio di due particelle viene tradotto
nello scambio di colonne della matrice di Slater, operazione
che comporta un cambiamento di segno del determinante stesso.

Nel caso in cui due particelle (1 e 2) occupassero lo
stesso spinorbitale $\psi_{{i}_1}$ il determinante rappresentativo
sarebbe
\beqa
\detsl{\psi_{{i}_1} \psi_{{i}_1} \ldots \psi_{{i}_n}} =
(n!)^{-\frac{1}{2}}
\left|
\begin{array}{cccc}
\psi_{{i}_1} (1) & \psi_{{i}_1} (1) & \ldots & \psi_{{i}_n} (1) \\
\psi_{{i}_1} (2) & \psi_{{i}_1} (2) & \ldots & \psi_{{i}_n} (2) \\
\vdots           &   \vdots         & \ddots &  \vdots          \\
\psi_{{i}_1} (n) & \psi_{{i}_1} (n) & \ldots & \psi_{{i}_n} (n) \\
\end{array}
\right|
\eeqa
che \`e nullo dal momento che presenta due colonne uguali. Il principio di
esclusione di Pauli \`e diretta conseguenza del principio di antisimmetria.

