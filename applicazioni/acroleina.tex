\subsection{Acroleina}

Per ultimo, abbiamo analizzato il sistema acroleina, appartenente al gruppo 
$C_s$.
\begin{center}
\begin{picture}(1200,900)(0,0)
\put(0,0){\Rtrigonal{0==C;1D==O;2==H}}
\put(171,206){\ethylene{1==C;2==C}{1==H;2==;3==H;4==H}}
\end{picture}
\end{center}
Il calcolo per il ground state, condotto su 28 configurazioni, ha
fornito un'energia di -190.878027 Hartree e una geometria come da tabella
\ref{tab:acroleina_geom}
\begin{center}
\begin{threeparttable}
\caption{Acrolein Ground State and Excited Geometry}
\label{tab:acroleina_geom}
\small
\begin{tabular}{|ccccc|}
\hline
							& GS			& Exper. GS \tnote{1}	& $n \rightarrow \pistar$	& $n \rightarrow \pistar$ (Exp.) \\
\hline
$r$(C=O)					& 1.219474		& 1.219 				& 	1.350912				& 1.344 \\
$r$(C-C)					& 1.478694		& 1.470					& 	1.373119				& 1.391 \\	
$r$(C=C)					& 1.319556		& 1.345					& 	1.399364				& 1.410 \\
$\angle$(O=C-C)				& 123.291 		& 123.3					& 	122.644					& 123.3 \\	
$\angle$(C-C=C)				& 121.868		& 119.8					& 	123.983					& 123.6 \\	
\hline
\end{tabular}
\begin{tablenotes}
\tiny
 \item[1] Cfr. \cite{jcp-103-14-1995-5877}
\end{tablenotes}
\end{threeparttable}
\end{center}

L'energia dello stato Ground State alla geometria di equilibrio e' risultata pari a -190.878026.
La transizione verticale $n_x \rightarrow \pistar$ richiede 4.64 eV, contro un valore sperimentale
di ??. La transizione adiabatica ha invece fornito un risultato pari a 3.52 eV, contro uno 
sperimentale di 3.21 eV (Cfr \cite{jacs-121-36-1999-8376})
