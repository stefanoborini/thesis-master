\subsection{Acetaldeide}

Successivamente si e' analizzato il sistema acetaldeide. Appartenente al gruppo
$C_s$, l'acetaldeide ha richiesto 30 configurazioni per il livello \texttt{CAS-CI}
richiesto. Lo stato HF vede un riempimento su 10 spinorbitali di simmetria 
$A^\prime$ e 2 spinorbitali di simmetria $A^{\prime\prime}$.

Analogamente ai precedenti CAS-CI, lo spazio di core e' costituito da
8 $A^\prime$ e 1 $A^{\prime\prime}$, mentre lo spazio attivo prende 3 $A^\prime$ e
2 $A^{\prime\prime}$ con 6 elettroni.

Le geometrie finali, confrontate ai valori sperimentali, sono in tabella \ref{tab:acetaldeide_geom}
\begin{center}
\begin{threeparttable}
\caption{Acetone Ground State and Excited Geometry}
\label{tab:acetaldeide_geom}
\small
\begin{tabular}{|cccc|}
\hline
							& 6-311G*/CAS	& Exper.\tnote{1} & 6-311G*/CAS $n_x \rightarrow \pistar$ \\
\hline
$r$(C-O)					& 1.217892		& 1.213 	& 1.389869 \\
$r$(C-C)					& 1.502840		& 1.504		& 1.497634 \\	
$r$(C-H)					& 1.092077		& 1.106		& 1.078165 \\	
$\angle$(O-C-C)				& 124.007		& 124		& 114.112 \\	
$\angle$(O-C-H)				& 119.395		& 121.1		& 111.097 \\	
$\angle$(C-C-H)				& 116.598		& 114.9		& 120.477 \\	
\hline
\end{tabular}
\begin{tablenotes}
\tiny
 \item[1] Cfr. \cite{jms-550-551-2000-281}
\end{tablenotes}
\end{threeparttable}
\end{center}

L'energia dello stato Ground State a questa geometria e' -153.024830.
La transizione calcolata $n_x \rightarrow \pistar$, di simmetria $A^{\prime\prime}$,
ha energia 152.849282. La transizione implica quindi una differenza di energia pari
a 4.78 eV, contro un valore sperimentale di 4.28 eV (Cfr. \cite{cpl-241-0-1995-26}).

Per la transizione adiabatica, il valore calcolato e' -152.883379, con una energia
di transizione pari a 3.85 eV, contro uno sperimentale di 3.69 eV (Cfr. \cite{jpc-97-17-1993-4293})

