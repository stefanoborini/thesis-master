\subsection{Benzene}

La molecola di benzene, di simmetria D$_{6h}$ \`e stata caratterizzata nello
stato fondamentale e negli stati eccitati di singoletto appartenenti alle rappresentazioni
B$_{2u}$ e B$_{1u}$. A causa di imposizioni del programma \texttt{dalton},
si \`e lavorato con il sottogruppo D$_{2h}$.

Inizialmente si \`e condotto un calcolo esplorativo, su base 6-31G* con spazio
CAS 6/6 definito dagli orbitali $\pi$ e dai corrispondenti $\pistar$. Tale
calcolo ha fornito geometrie e valutazioni degli orbitali e delle loro
simmetrie.

In seguito, per affinare i risultati, si \`e provveduto ad effettuare altre
valutazioni, passando quindi alla base ano-1 definita 3s2p1d per il carbonio
e 2s1p per l'idrogeno, sia con il medesimo spazio CAS sopra descritto, sia
con uno spazio CAS ampliato, ottenuto raddoppiando gli orbitali per ogni
simmetria, e passando di conseguenza ad un CAS 6/12.

La molecola \`e stata posta sul piano $xy$, con l'asse principale orientato
lungo l'asse $z$.

I ben noti orbitali $\pi$ del benzene, orbitali che faranno parte dello
spazio attivo non espanso, in simmetria D$_{6h}$ sono assegnabili alle
rappresentazioni A$_{2u}$, E$_{1g}$, E$_{2u}$ e B$_{2g}$. In simmetria
D$_{2h}$ tali rappresentazioni si trasformano, rispettivamente e ricordando
che le rappresentazioni E sono bidimensionali, in B$_{1u}$, B$_{2g}$,
B$_{3g}$, B$_{1u}$, A$_{u}$ e B$_{2g}$. Gli orbitali appartenenti alle prime
tre rappresentazioni sono occupati.
Sempre in seguito alla diminuzione di simmetria, gli stati eccitati su cui
verranno effettuate valutazioni apparterranno, in simmetria D$_{2h}$,
alle rappresentazioni B$_{2u}$ e B$_{3u}$, anzich\'e B$_{2u}$ e B$_{1u}$,
rispettivamente, per la simmetria D$_{6h}$.

Le geometrie molecolari sono state ottimizzate nelle varie condizioni
operative. Nella tabella \ref{tab:benzene_geom} sono messe a confronto le
geometrie ottenute per via teorica con il dato sperimentale
\begin{center}
\begin{threeparttable}
\caption{\small Benzene - geometrie su diversi metodi/basi}
\label{tab:benzene_geom}
\small
\begin{tabular}{|l|c|c|}
\hline
							& R$_{CC}$		& R$_{CH}$ \\ 
\hline
CASSCF 6/6 / 6-31G*			& 1.396			&	1.075	 \\
CASSCF 6/6 / ano-1			& 1.398			&	1.080	 \\
CASSCF 6/12 / ano-1			& 1.398			&	1.081    \\
Exp.\tnote{1}				& 1.390			&	1.086    \\
\hline
\end{tabular}
\begin{tablenotes}
\small
 \item[1] Cfr. \cite{jms-148-1991-427}
 \item[] Valori in Angstroms
\end{tablenotes}
\end{threeparttable}
\end{center}

\subsubsection{Eccitazione B$_{2u}$ (B$_{2u}$ in D$_{2h}$)}

La transizione elettronica B$_{2u}$ \`e quella a minore energia. Analisi
sperimentali (Cfr. \cite{jcp-94-12-1991-7700}) forniscono per questa
transizione un valore di 4.90 eV, anche se in realt\`a lo spettro in fase gas
e jet-cooled forniscono un inviluppo vibrazionale che va da 4.79 eV
(transizione 0-0) fino a 5.35 eV, con un picco di massimo assorbimento
proprio a 4.90 eV, scelto come riferimento per la transizione Franck-Condon.

La tabella \ref{tab:benzene_b2u} mostra i risultati ottenuti con la
trattazione effettuata.
\begin{center}
\begin{threeparttable}
\caption{\small Benzene - Energia della transizione di singoletto B$_{2u}$
(B$_{2u}$ in D$_{2h}$)}
\label{tab:benzene_b2u}
{
\small
\begin{tabular}{|c|ccc|ccc|}
\hline
 				& \multicolumn{3}{c|}{GS\tnote{1}}				& \multicolumn{3}{c|}{Ecc. B$_{2u}$\tnote{2}} \\
				& CASSCF	& NEV-PT	& NEV-PT	& CASSCF		& NEV-PT	& NEV-PT \\
				&			& SC		& PC		& 				& SC		& PC \\
\hline
CAS(6,6)/6-31G*	& 0.775778	& 1.468765	& 1.469361	& 4.99			& 5.33		& 5.31		\\
CAS(6,6)/ano-1	& 0.830416	& 1.558968	& 1.559728	& 4.92			& 5.22		& 5.20		\\
CAS(6,12)/ano-1	& 0.844274	& 1.558662	& 1.564030	& 4.92			& 5.17		& 5.19		\\
\hline
\hline
Exp.\tnote{3}	&				& 				&				& \multicolumn{3}{c|}{4.90} \\
\hline
\end{tabular}
}
\begin{tablenotes}
\small
 \item[1] Energia come \mbox{-(230 + valore)} Hartree
 \item[2] Valori in eV
 \item[3] Cfr. \cite{jcp-94-12-1991-7700}
\end{tablenotes}
\end{threeparttable}
\end{center}

Apparentemente quindi, il risultato CASSCF \`e in buon accordo con il dato
sperimentale, e la trattazione perturbativa attua un allontanamento dal
valore esatto. Tuttavia, la Ref. \cite{jcp-112-6-2000-2798} sottolinea come
la differenza tra la transizione 0-0 ed il massimo della banda sia 0.19 eV,
mentre sui valori calcolati la differenza tra la transizione 0-0 e la
verticale \`e di 0.31 eV. Di conseguenza, il massimo della banda \`e spostato
di 0.12 eV pi\`u in basso rispetto all'energia verticale Franck-Condon.
Questo implica che il valore sperimentale di 4.90 eV comporta, per la
transizione verticale, un posizionamento a 5.02 eV, in migliore accordo
con i dati ottenuti a livello perturbativo.

\subsubsection{Eccitazione B$_{1u}$ (B$_{3u}$ in D$_{2h}$)}

La transizione B$_{1u}$ viene stimata attorno ai 6.20 eV (Cfr.
\cite{jcp-112-6-2000-2798} e \cite{tca-91-1995-91}). La tabella
\ref{tab:benzene_b1u} mostra i dati ottenuti
\begin{center}
\begin{threeparttable}
\caption{\small Benzene - Energia della transizione di simmetria B$_{1u}$
(B$_{3u}$ in D$_{2h}$)}
\label{tab:benzene_b1u}
{
\small
\begin{tabular}{|c|ccc|}
\hline
 				& \multicolumn{3}{c|}{ Ecc. B$_{1u}$\tnote{2}} \\
				& CASSCF		& NEV-PT/SC & NEV-PT/PC \\
\hline
CAS(6,6)/6-31G*	&  8.18			& 6.75		& 6.71		\\
CAS(6,6)/ano-1	&  7.87			& 6.33 		& 6.26		\\
CAS(6,12)/ano-1	&  7.46			& 6.54		& 6.50		\\
\hline
\hline
Exp.\tnote{3}	&	 \multicolumn{3}{c|}{6.20} \\
\hline
\end{tabular}
}
\begin{tablenotes}
\small
 \item[1] Valori in Hartree
 \item[2] Valori in eV
 \item[3] Cfr. \cite{jcp-112-6-2000-2798} e \cite{tca-91-1995-91}
\end{tablenotes}
\end{threeparttable}
\end{center}
