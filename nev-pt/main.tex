\chapter{Teoria perturbativa NEV-PT}

\section{Introduzione}

\`E stato possibile vedere come la migliore accuratezza possibile per
un dato sistema sia ottenibile, in linea di principio, con una espansione
Full-CI su una base data. Questo approccio \`e percorribile con difficolt\`a 
a causa della complessit\`a computazionale eccessiva. Per porre
rimedio a tale limitazione, definire uno schema di selezione per gli orbitali
pu\`o essere una buona approssimazione, se tale schema prende in
considerazione gli orbitali maggiormente significativi.
Questo approccio multireference pu\`o essere eventualmente migliorato 
applicando una correzione perturbativa alle funzioni d'onda e alle relative
energie.

Le teorie perturbative multireference (MRPT) costituiscono uno strumento
prezioso per il calcolo dell'energia di correlazione in molecole di piccole
e medie dimensioni. Sebbene i primi tentativi di formulare teorie MRPT
risalgano ad oltre trent'anni fa (Cfr. \cite{rmp-39-1967-771},
\cite{jcp-58-1973-5745}) \`e solamente nell'ultima decade che tali strumenti
sono diventati di utilizzo comune. Lo scopo delle teorie perturbative
multireference \`e di fornire uno strumento agile e preciso, che ricalchi il
famoso approccio M{\o}ller-Plesset (Cfr. \cite{pr-46-1934-618}) per il caso
di una funzione d'onda monodeterminantale.
Come \`e noto, nel caso di una molecola che sia ben descritta da un singolo
determinante di Slater (come la maggior parte delle molecole nello stato
fondamentale ed alla geometria di equilibrio) l'applicazione della teoria
perturbativa al secondo ordine nell'energia, nell'approccio
M{\o}ller-Plesset (MP2) fornisce spesso oltre il 90 \% dell'energia di
correlazione.
L'hamiltoniano di ordine zero della teoria MP \`e molto semplice, e consiste
nell'operatore di Fock a $n$ particelle
\beq
\ham_0 = \sumonen{i} \fock(i)
\eeq
Applicando quanto esposto nella sezione \ref{sec:perturbazione} si ottiene
subito, al secondo ordine
\beq
E_0^{(2)} = - \sum_{i,j}^{\mbox{occ}} \sum_{r,s}^{\mbox{virt}} \frac{\left|
\interact{rs}{ij}\right|^{2}}{\epsilon_r + \epsilon_s - \epsilon_i -
\epsilon_j }
\eeq

Nel caso di una MRPT, la funzione d'onda di ordine zero \`e multireference,
ossia una combinazione lineare di determinanti di Slater. Di conseguenza,
non \`e ovvio come si possa arrivare ad una efficace definizione
dell'hamiltoniano di ordine zero. I vari metodi MRPT rientrano in due
distinte categorie: la prima viene indicata come "\textit{perturb then diagonalize}"
e si basa sulla possibilit\`a di costruire un hamiltoniano efficace,
definito perturbativamente, che sar\`a diagonalizzato (Cfr. \cite{rmp-39-1967-771} e
\cite{jpa-18-1985-809})

La seconda categoria, nota invece come "\textit{diagonalize then perturb}",
parte da una funzione d'onda variazionale ed applica a questa la teoria
perturbativa. A questa categoria appartengono le tecniche CIPSI, iniziate a
Parigi da Huron \textit{et al.} (Cfr. \cite{jcp-58-1973-5745}) e
successivamente proseguite a Pisa (Cfr. \cite{jcc-8-1987-39}) e 
Ferrara (Cfr. \cite{ijqc-60-1996-167}, \cite{tca-98-1997-57} e
\cite{tca-98-1997-117}), le quali si basano sull'uso
dei semplici determinanti di Slater come funzioni perturbatrici, e la
tecnica CASPT2 (Cfr. \cite{jpc-94-1990-5483} e \cite{jcp-96-1992-1218}) che
utilizza invece funzioni "contratte" (particolari combinazioni di
determinanti) nella teoria perturbativa.

Tra le propriet\`a pi\`u desiderabili di una teoria perturbativa
multireference citiamo le seguenti:
\begin{itemize}
\item size consistence (o strict separability). Si vorrebbe che nel calcolo
di un sistema consistente di due parti non interagenti (AB) l'energia
risultasse uguale alla somma delle due parti (E$_{AB} = $E$_A + $E$_B$)
\item assenza di "intruder states". Le funzioni che correggono la funzione
d'onda all'ordine zero dovrebbero avere energie ben distinte da quella della
funzione d'ordine zero stessa, in maniera da evitare divergenza. La semplice
formula di M{\o}ller-Plesset soddisfa questo requisito: le funzioni
perturbatrici sono i semplici determinanti di doppia sostituzione
$\Phi_{ij}^{rs}$ che, data la forma adottata per $\ham_0$ hanno come energia
$E_0 + \epsilon_r + \epsilon_s - \epsilon_i - \epsilon_j$, ben distinta da
$E_0$.
\item L'accuratezza nel calcolo degli stati elettronicamente eccitati
dovrebbe essere comparabile con quella ottenuta per lo stato fondamentale.
\end{itemize}

Per svariati motivi, le teorie perturbative fin qui citate falliscono
nell'esaudire uno o pi\`u di questi requisiti.
L'approccio perturbativo $n$-Electron Valence Perturbation Theory (NEV-PT)
(Cfr. \cite{jcp-114-23-2001-10252}), sviluppato in collaborazione tra il
nostro gruppo di Ferrara e quello del Prof. Malrieu dell'universit\`a di
Tolosa, risolve in modo formale sia le necessit\`a di size-consistency, sia
il problema di stati intrusi, che sono le principali necessit\`a di una
teoria perturbativa, ed inoltre si comporta come un approccio
M{\o}ller-Plesset se applicato ad una funzione
single reference.

Nello sviluppo dell'approccio NEV-PT si \`e giocato su due variabili:
\begin{itemize}
\item il grado di contrazione dello spazio perturbatore
\item la scelta dell'hamiltoniano
\end{itemize}

In alcuni casi, si \`e fatto uso dell'hamiltoniano proposto da Dyall (Cfr.
\cite{jcp-102-1995-4909}). Tale hamiltoniano risolve il problema formale che
nasce nel momento in cui la funzione d'onda multireference all'ordine zero
incorpora effetti di interazione bielettronica, mentre qualsiasi trattazione
perturbativa che si rifaccia all'hamiltoniano M{\o}ller-Plesset fa uso di
un hamiltoniano puramente monoelettronico.
L'Hamiltoniano di Dyall \`e un hamiltoniano approssimato bielettronico, che
quindi si presta meglio, dal punto di vista formale, alle trattazioni a
seguire.

\section{Teoria}

La teoria perturbativa NEV-PT necessita di una funzione all'ordine zero
definita su uno spazio CAS
\beq
\Psi_m^{(0)} = \sumidx{I \in \mbox{CAS}} C_{I,m} \ket{I}
\eeq
ottenuto diagonalizzando $\ham$ nello spazio CAS
\beq
\proj{\mbox{CAS}}\ham\proj{\mbox{CAS}}\ket{\Psi_m^{(0)}} = E_m^{(0)} \ket{\Psi_m^{(0)}}
\eeq
dove $\proj{CAS}$ \`e il proiettore all'interno dello spazio CAS
\beq
\proj{\mbox{CAS}} = \sum_{M}^{\mbox{CAS}} \ket{\mbox{M}} \bra{\mbox{M}}
\eeq
Le funzioni perturbatrici sono esterne a tale spazio, e quelle di interesse
per il metodo NEV-PT sono classificabili nel seguente schema: la funzione
d'onda all'ordine zero \`e scrivibile come un prodotto antisimmetrizzato di
una parte inattiva con $n_c$ elettroni e di una funzione
multireference di valenza con $n_v$ elettroni
\beq
\ket{\Psi_m^{(0)}} = \ket{\Phi_c \Psi_m^v}
\eeq
e in modo analogo le funzioni perturbatrici avranno la forma
\beq
\ket{\Phi_l^{-k} \Psi_{\mu}^{v+k}}
\eeq
dove $k$ \`e il numero di elettroni rimossi dallo spazio inattivo e
introdotti nello spazio attivo ed $l$ denota gli orbitali inattivi (core + virtuale).
Al secondo ordine di perturbazione $k$ \`e
compreso tra -2 e 2.
Richiediamo inoltre che non esista interazione tra due funzioni
perturbatrici appartenenti alla medesima classe, individuata dagli indici
$k$ e $l$. In altri termini, deve valere
\beq
\braket{\Phi_l^{-k} \Psi_{\mu}^{v+k}}{\ham}{\Phi_l^{-k} \Psi_{\nu}^{v+k}} =
0 \qquad \mbox{per }\mu \neq \nu
\eeq

Il modo migliore per definire tali funzioni perturbatrici \`e espanderle su
una base di uno spazio $S_l^k$  individuato dai determinanti con la
medesima parte inattiva $\Phi_l^{-k}$ e tutte le possibili parti attive
$\Phi_I^{+k}$. In altri termini, tale spazio avr\`a come base il set di
determinanti
\beq
\left\{ \Phi_l^{-k} \Phi_I^{+k} \right\}
\eeq
e le funzioni perturbatrici saranno combinazioni lineari opportune
all'interno di tale spazio. Questo tipo di approccio sfrutta l'intera
dimensionalit\`a offerta dallo spazio perturbatore $S_l^k$ e viene denominato
\textbf{totalmente non contratto}.

Per ottenere le funzioni pertubatrici sar\`a sufficiente diagonalizzare
l'hamiltoniano all'interno di tale spazio
\beq
\label{eqn:teoria1_1}
\proj{S_l^k}\ham\proj{S_l^k} \ket{\Phi_l^{-k} \Psi_{\mu}^{v+k}} = E_{l,\mu}
\ket{\Phi_l^{-k} \Psi_{\mu}^{v+k}}
\eeq
tuttavia tale procedura \`e computazionalmente pesante. Una possibile
semplificazione si ottiene con l'utilizzo dell'hamiltoniano di Dyall,
che \`e definito nel modo seguente
\beqa
\ham^D &=& \ham^D_i + \ham^D_v \\
\ham^D_i &=& \sumidx{i} \epsilon_i \constr{i} \destr{i} + \sumidx{r}
\epsilon_r \constr{r} \destr{r} + C \\
\ham^D_v &=& \sumidx{ab} h_{ab}^{\mbox{eff}} \constr{a} \destr{b} +
\frac{1}{2} \sumidx{abcd} \integral{ab}{cd} \constr{a}\constr{b}
\destr{d}\destr{c}
\eeqa
dove con indici $i,j,\ldots$, $a,b,\ldots$, $r,s,\ldots$ si indicano
rispettivamente orbitali di core, attivi e virtuali e definiamo 
$ h_{ab}^{\mbox{eff}} = \braket{a}{h+\sumidx{i}\left( J_i - K_i \right)}{b}$.

$\ham^D$ gode della propriet\`a di comportarsi esattamente come l'hamiltoniano
vero all'interno dello spazio CAS, dopo un'appropriata scelta della costante C.
Come conseguenza, tale hamiltoniano possiede gli stessi autovalori e gli stessi
autovettori dell'hamiltoniano vero proiettato all'interno dello spazio CAS.

%La struttura dell'hamiltoniano di Dyall permette inoltre di considerare che,
%nell'equazione \ref{eqn:teoria1_1} relativa a questo hamiltoniano, le energie
%$E_{l,\mu}$ possono essere ricavate da una traslazione energetica
%appropriata (e determinata esclusivamente dalla distribuzione elettronica
%della parte inattiva) rispetto al valore $E_{\mu,k}$ restituito dalla
%relazione
%\beq
%\ham_v^D \ket{ {\Psi^{\prime}}_{\mu}^{v+k}} = E_{l,\mu} \ket{
%{\Psi^{\prime}}_{\mu}^{v+k}}
%\eeq
%che interessa esclusivamente la parte attiva, in quanto la parte inattiva
%pu\`o essere operata a parte in virt\`u della forma dell'operatore di Dyall,
%e comporta appunto una traslazione energetica, come enunciato poco sopra.

\subsection{Approccio partially contracted}

L'approccio sopra enunciato fa uso dell'intera dimensionalit\`a degli spazi
$S_l^k$. Una alternativa computazionalmente pi\`u semplice \`e l'utilizzo di
un sottospazio di $S_l^k$, definito dall'applicazione di un operatore di perturbazione
$\perturb$ alle funzioni d'onda CAS $\Psi_m^{(0)}$. \`E possibile
dimostrare che l'operatore di perturbazione \`e scrivibile come una somma di
contributi
\beqa
\perturb^u &=& \perturb^{(0)} + \perturb^{(+1)} + \perturb^{(-1)} +
\perturb^{(+2)} + \perturb^{(-2)} + \perturb^{\prime(+1)} \nonumber \\
& & + \perturb^{\prime(-1)} + \perturb^{\prime(0)}
\eeqa

Ciascun contributo \`e costituito da una somma di termini dell'hamiltoniano,
i quali, applicati a $\Psi_m^{(0)}$, generano funzioni appartenenti ai vari
spazi $S_l^k$. Nel complesso, tali funzioni generano sottospazi di
$S_l^k$, che indicheremo con $\overline{S}_l^k$.

Ad esempio, l'operatore $\perturb^{(+1)}$, che rappresenta la componente
perturbativa il cui effetto \`e di promuovere un elettrone dagli orbitali di
core a quelli attivi e un altro elettrone dagli orbitali di core ai
virtuali, sar\`a
\beq
\perturb^{(+1)} = \sumidx{i<j} \sumidx{r} \sumidx{a} \interact{ra}{ji}
\constr{r}\constr{a}\destr{i}\destr{j} = \sumidx{i<j}\sumidx{r}
\perturb_{ijr}^{(+1)}
\eeq
e ciascuno degli elementi di $\perturb_{ijr}^{(+1)}$ genera funzioni del
tipo $\ket{\Phi_l^{-1} \constr{a} \Psi_{m}^{v}}$ che appartengono allo spazio
$S_l^{(+1)}$. Il set di funzioni $\left\{ \ket{\Phi_l^{-1} \constr{a} \Psi_{m}^{v}} \right\}$
genera uno spazio $\overline{S}_l^{+1}$ la cui dimensionalit\`a \`e uguale
al numero degli orbitali attivi. La diagonalizzazione dell'hamiltoniano
completo o di Dyall in ciascuno di questi sottospazi fornisce le funzioni
perturbatrici necessarie, ottimizzate variazionalmente, e le relative
energie. Tale risoluzione pu\`o essere applicata a qualsiasi componente sopra
presentato, e generer\`a funzioni che apparterranno ad opportuni sottospazi.

Una particolarit\`a \`e fornita dall'operatore $\perturb_{ijrs}^{(0)}$. Lo
spazio ridotto $\overline{S}_l^{0}$, per una data scelta degli indici $i,j,r,s$,
contiene una sola funzione
significativa per il calcolo perturbativo, che descrive una doppia eccitazione
dallo spazio di core allo spazio virtuale. 
L'energia associata a tale funzione perturbatrice quando si adoperi
l'hamiltoniano di Dyall non \`e altro che l'energia della funzione CAS
corretta con l'aggiunta del termine $\epsilon_r + \epsilon_s - \epsilon_i -\epsilon_j$,
ovvero la differenza delle energie degli orbitali interessati all'eccitazione.

La correzione perturbativa al primo ordine della funzione d'onda
per tale caso sar\`a data da
\beqa
\Psi_{m}^{(1)}(\perturb^{(0)}) &=& - \sumidx{i<j} \sumidx{r<s} 
\ket{\constr{r}\constr{s}\destr{i}\destr{j} \Phi_{c} \Psi_m^v} \nonumber \\
& & \times \frac{\braket{\constr{r}\constr{s}\destr{i}\destr{j} \Phi_{c}
\Psi_m^v}{\ham}{\Psi_m^{(0)}}}{\epsilon_r + \epsilon_s - \epsilon_i
- \epsilon_j}
\eeqa

e conseguentemente la correzione al secondo ordine per l'energia sar\`a data
dal valore
\beq
\braket{\Psi_{m}^{(0)}}{\ham}{\Psi_{m}^{(1)}(\perturb^{(0)})}
\eeq
ovvero
\beq
E_{m}^{(2)}(\perturb^{(0)}) = - \sumidx{i<j} \sumidx{r<s} \frac{\left| \left< rs \left|
\right| ji \right> \right|^2}{\epsilon_r + \epsilon_s -
\epsilon_i-\epsilon_j}
\eeq

che non \`e altro che il contributo M{\o}ller-Plesset MP2, nel caso di una funzione HF.
La teoria NEV-PT di conseguenza si riduce ad una trattazione M{\o}ller-Plesset
nel momento in cui lo spazio attivo non esista (come nel caso
dell'approssimazione monodeterminantale,
a cui la perturbazione MP2 fa riferimento).

Nel caso dei termini di $\perturb$ che coinvolgano lo spazio attivo, la
trattazione \`e pi\`u complessa. Le funzioni saranno del tipo $\ket{\constr{r}
\destr{i}\destr{j}\Phi_c \Psi_{\mu}^{(+1)}}$, dove la parte attiva \`e
caratterizzata da un elettrone in pi\`u rispetto alla funzione CAS.
La correzione alla funzione d'onda al primo ordine sar\`a
\beqa
\Psi_{m}^{(1)}(\perturb^{(+1)}) &=& - \sumidx{i<j} \sumidx{r} \sumidx{\mu}
\ket{\constr{r}\destr{i}\destr{j}\Phi_c \Psi_{\mu}^{(+1)}} \nonumber \\ 
&\times& \frac{\braket{\constr{r}\destr{i}\destr{j}\Phi_c \Psi_{\mu}^{(+1)}}{\ham}{\Psi_m^{(0)}}}{E_{\mu}^{(+1)} - E_m^{(0)} + \epsilon_r - \epsilon_i - \epsilon_j}
\eeqa

Si dovr\`a quindi calcolare una interazione del tipo
\beq
\braket{\constr{r}\destr{i}\destr{j}\Phi_c\Psi_{\mu}^{(+1)}}
{\ham}{\Psi_m^{(0)}}
\eeq

Siccome l'interazione avviene esclusivamente attraverso
il termine perturbatore, risulter\`a
\beq
\braket{\constr{r}\destr{i}\destr{j}\Phi_c\Psi_{\mu}^{(+1)}} {\perturb^{(+1)}_{ijr}}{\Psi_m^{(0)}}
\eeq
dalla quale, per svolgimento dell'operatore e dopo alcuni passaggi, si
otterr\`a
\beqas
\integral{\constr{r}\destr{i}\destr{j}\Phi_c\Psi_{\mu}^{(+1)}}{\sumidx{a}\interact{ra}{ji}\constr{r}\destr{i}\destr{j}\Phi_c\constr{a}\Psi_m^v} = 
\eeqas
\beqa
&=& \sumidx{a} \interact{ra}{ji} \integral{\Psi_\mu^{(+1)}}{\constr{a}\Psi_m^v} \nonumber \\
&=& \sumidx{a} \interact{ra}{ji} S_{\mu a} \stackrel{\mbox{\tiny def}}{=}
\interact{r\mu}{ji}
\eeqa

Nella formula precedente, il termine $\sumidx{a} \interact{ra}{ji} S_{\mu a}$ \`e stato ridotto
ad un solo integrale bielettronico efficace che coinvolge una ``quasi-particella'' nella valenza.
Di conseguenza, tale termine pu\`o essere scritto tramite il semplice integrale
$\interact{r\mu}{ji}$ e la correzione perturbativa al secondo ordine 
sull'energia sar\`a
\beq
E_{(+1),m}^{(2)} = - \sumidx{i<j} \sumidx{r} \sumidx{\mu}
\frac{\left| \interact{r\mu}{ji} \right| ^2}{E^{(+1)}_{\mu} - E_m^{(0)} + \epsilon_r -
\epsilon_i - \epsilon_j }
\eeq

Analogamente, gli altri contributi rappresentano l'inserimento di
``quasi-buche'' o ``quasi-coppie'' all'interno dello strato di valenza.
Si otterranno quindi una serie di contributi all'energia per la correzione
perturbativa, che saranno centrali nell'analisi dei dati attuata nel
seguito di questa tesi.

\subsection{Approccio strongly contracted}

Un'ulteriore semplificazione pu\`o essere introdotta riducendo ancora la 
dimensione dello spazio perturbativo. Ogni termine $\hat{\mathcal{V}}$,
come \`e stato visto in precedenza, genera un sottospazio perturbativo la
cui dimensione dipende in modo diretto da quella dello spazio attivo
selezionato. \`E tuttavia possibile utilizzare gli operatori perturbativi in
modo da fornire una sola funzione appartenente a ciascuno
spazio, ed attuare la trattazione esclusivamente con tali funzioni.

Al fine di ottenere ci\`o, nel caso di esempio $\perturb^{(+1)}$, avremo
\beq
\ket{\phi_{ijr}} = \perturb_{ijr}^{(+1)} \ket{\Psi_m^{(0)}} = \constr{r}
\destr{i} \destr{j} \Phi_c \sumidx{a}\interact{ra}{ji} \constr{a}
\Psi^{v}_{m}
\eeq

Questa funzione d'onda, generalmente non normalizzata, definisce uno spazio
monodimensionale $\tilde{S}_l^k$ che sar\`a considerato al fine della
trattazione perturbativa per la parte $\perturb^{(+1)}$.
La norma di tale funzione assume un significato molto importante, in quanto
\`e direttamente implicata nel trattamento perturbativo, e pu\`o facilmente
essere ottenuta da:
\beqa
N_{ijr} = \integral{\phi_{ijr}}{\phi_{ijr}} &=& \left( \sumidx{a} \interact{ra}{ji}
\ket{\constr{a}\Psi_m^{v}} \right) \left( \sumidx{b} \interact{rb}{ji}
\ket{\constr{b}\Psi_m^v} \right) \nonumber \\
&=& \sumidx{ab} \interact{ra}{ji} \integral{\constr{a} \Psi_m^v}{\constr{b}
\Psi_m^v} \interact{rb}{ji} \nonumber \\
&=& \sumidx{ab} \interact{ra}{ji} \tilde{\rho}_{ba} \interact{rb}{ji}
\eeqa
dove $\tilde{\rho}_{ba} =
\braket{\Psi_m^{(0)}}{\destr{a}\constr{b}}{\Psi_m^{(0)}}$ \`e la matrice
densit\`a ad una buca (legata alla matrice densit\`a ad una particella dalla
relazione $\tilde{\rho}_{ba} = \delta_{ab} - \rho_{ab}$)

Quindi la funzione d'onda normalizzata sar\`a data da 
\beq
\ket{\phi_{ijr}^{\prime}} = \frac{1}{\sqrt{N_{ijr}}} \ket{\phi_{ijr}}
\eeq

Da questa \`e possibile ricavare i coefficienti della correzione perturbativa
al primo ordine:
\beqa
C_{ijr}^{(1)} &=& \frac{\braket{\phi^{\prime}_{ijr}}{\ham}{\Psi_m^{(0)}}}{E_m^{(0)} - E_{ijr}} \nonumber \\
&=& \frac{\braket{\phi^{\prime}_{ijr}}{\perturb_{ijr}^{(+1)}}{\Psi_m^{(0)}}}{E_m^{(0)} - E_{ijr}} \nonumber \\
&=& \frac{\integral{\phi^{\prime}_{ijr}}{\phi_{ijr}}}{E_m^{(0)} - E_{ijr}} \nonumber \\
&=& \frac{\sqrt{N_{ijr}}}{E_m^{(0)} - E_{ijr}}
\eeqa
e di conseguenza la correzione perturbativa all'energia
\beq
E_m^{(2)} = \frac{N_{ijr}}{E_m^{(0)} - E_{ijr}}
\eeq
dove il termine $E_{ijr}$ \`e il valore medio dell'hamiltoniano di Dyall
nella funzione $\ket{\phi^{\prime}_{ijr}}$.

