\section{Dimensionalita' dello spazio CAS}

Come detto, per attuare una correzione all'energia HF di una molecola, e' necessario
considerare la funzione d'onda CAS costituita partizionando il set di spinorbitali
in tre insiemi, core, attivi e virtuali, e di attuare uno sviluppo multiconfigurazionale
negli orbitali attivi. Portando al limite tale espansione, si raggiunge la massima 
accuratezza possibile sul set di spinorbitali e sulla dimensionalita' scelta, 
e si parla in tal caso di Full CI.

La dimensione di uno spazio CAS e' espressione combinatoriale del numero di elettroni $n_a$
e del numero di orbitali attivi $N_a$. Trattandosi di una combinazione semplice di $N_a$ oggetti
di classe $n_a$, risulta che lo spazio CAS e' dato dal coefficiente binomiale 
\beqas
\left( 
\begin{array}{c}
N_a \\
n_a
\end{array}
\right)
= \frac{N_a !}{n_a ! \left( N_a - n_a \right)!}
\eeqas

Trascurando eventuali simmetrie, tale spazio diventa computazionalmente non trattabile
gia' su sistemi piccoli, e cio' limita l'applicabilita' di tale metodo. Inoltre, 
l'utilizzo di uno spazio CAS consente solamente la corretta descrizione dell'andamento
dissociativo (correlazione non dinamica) ma non consente accuratezza spettroscopica,
in quanto gli effetti di correlazione dinamica vengono trascurati, fornendo una curva
energetica traslata rispetto alla curva sperimentale.
Per tenere conto di tali effetti, occorrerebbe procedere all'inclusione di termini
almeno singolarmente e doppiamente eccitati rispetto al livello MC, e questo comporta
un ulteriore aumento della complessita' computazionale.

Una possibile soluzione a questi inconvenienti e' considerare una trattazione
localizzata: attraverso una opportuna trasformazione unitaria degli orbitali
delocalizzati MO, si ottiene un nuovo set che descrive la molecola in termini
di:
\begin{itemize}
\item Determinanti neutri A$^{(0)}\ldots$B$^{(0)}$
\item Determinanti singolarmente ionici A$^{(+1)}\ldots$B$^{(-1)}$ o A$^{(-1)}\ldots$B$^{(+1)}$ 
\item Determinanti multiplamente ionici A$^{(+p)}\ldots$B$^{(-p)}$ o A$^{(-p)}\ldots$B$^{(+p)}$
\end{itemize}
E' logico aspettarsi che i determinanti neutri e i singolarmente ionici siano quelli
a maggior peso nella rappresentazione della molecola, in quanto sono quelli a minore
energia. Per questa ragione, sara' possibile (entro certi limiti) trascurare i contributi
multiplamente ionici riducendo in questo modo la complessita' computazionale del problema.



