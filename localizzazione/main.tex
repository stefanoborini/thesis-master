\chapter{Trattazione teorica della localizzazione}

Come descritto in precedenza (v. \ref{subsec:idrogeno}, pag. \pageref{subsec:idrogeno}) e' possibile
interpretare il legame chimico in due differenti modalita': orbitali molecolari o legame di valenza.
Ciascuna delle due interpretazioni ha punti di forza e di debolezza, e l'una o l'altra visione
e' valida per la rappresentazione di una molecola: 

\begin{itemize}
\item nella visione MO, abbiamo combinazioni lineari di rappresentazioni, ciascuna delle quali 
rappresenta diverse distribuzioni elettroniche in orbitali molecolari; 

\item nella visione VB, abbiamo combinazioni di formule di risonanza, ciascuna delle quali ha caratteristiche
di neutralita' o ionicita'.
\end{itemize}

In entrambe le visioni, e' presumibile che le configurazioni a maggiore energia (ovvero quelle a
maggiore eccitazione nel caso MO, quelle a maggiore ionicita' nel caso VB) abbiano un peso minore
delle configurazioni a piu' bassa energia, ma tale peso diviene tuttavia non trascurabile se si
tiene conto del numero di queste configurazioni, che forniscono comunque un contributo non nullo
all'energia globale del sistema, sotto forma di energia di correlazione. Tale contributo, per
il teorema variazionale, tende a far abbassare l'energia del sistema, in quanto la funzione
d'onda finale e' meglio descritta mano a mano che si tiene conto del contributo di correlazione.

Logicamente, deve esistere una qualche forma di legame tra le due visioni: il successo
della teoria MO deriva dalla sua facile adattabilita' ai sistemi computazionali, e alla diretta
correlazione con una grandezza fisicamente tangibile (anche se approssimativa) quale l'energia
di ionizzazione e l'affinita' elettronica attraverso il teorema di Koopmans semplice ed esteso.

\section{Dimensionalita' dello spazio CAS}

Come detto, per attuare una correzione all'energia HF di una molecola, e' necessario
considerare la funzione d'onda CAS costituita partizionando il set di spinorbitali
in tre insiemi, core, attivi e virtuali, e di attuare uno sviluppo multiconfigurazionale
negli orbitali attivi. Portando al limite tale espansione, si raggiunge la massima 
accuratezza possibile sul set di spinorbitali e sulla dimensionalita' scelta, 
e si parla in tal caso di Full CI.

La dimensione di uno spazio CAS e' espressione combinatoriale del numero di elettroni $n_a$
e del numero di orbitali attivi $N_a$. Trattandosi di una combinazione semplice di $N_a$ oggetti
di classe $n_a$, risulta che lo spazio CAS e' dato dal coefficiente binomiale 
\beqas
\left( 
\begin{array}{c}
N_a \\
n_a
\end{array}
\right)
= \frac{N_a !}{n_a ! \left( N_a - n_a \right)!}
\eeqas

Trascurando eventuali simmetrie, tale spazio diventa computazionalmente non trattabile
gia' su sistemi piccoli, e cio' limita l'applicabilita' di tale metodo. Inoltre, 
l'utilizzo di uno spazio CAS consente solamente la corretta descrizione dell'andamento
dissociativo (correlazione non dinamica) ma non consente accuratezza spettroscopica,
in quanto gli effetti di correlazione dinamica vengono trascurati, fornendo una curva
energetica traslata rispetto alla curva sperimentale.
Per tenere conto di tali effetti, occorrerebbe procedere all'inclusione di termini
almeno singolarmente e doppiamente eccitati rispetto al livello MC, e questo comporta
un ulteriore aumento della complessita' computazionale.

Una possibile soluzione a questi inconvenienti e' considerare una trattazione
localizzata: attraverso una opportuna trasformazione unitaria degli orbitali
delocalizzati MO, si ottiene un nuovo set che descrive la molecola in termini
di:
\begin{itemize}
\item Determinanti neutri A$^{(0)}\ldots$B$^{(0)}$
\item Determinanti singolarmente ionici A$^{(+1)}\ldots$B$^{(-1)}$ o A$^{(-1)}\ldots$B$^{(+1)}$ 
\item Determinanti multiplamente ionici A$^{(+p)}\ldots$B$^{(-p)}$ o A$^{(-p)}\ldots$B$^{(+p)}$
\end{itemize}
E' logico aspettarsi che i determinanti neutri e i singolarmente ionici siano quelli
a maggior peso nella rappresentazione della molecola, in quanto sono quelli a minore
energia. Per questa ragione, sara' possibile (entro certi limiti) trascurare i contributi
multiplamente ionici riducendo in questo modo la complessita' computazionale del problema.




