\pagebreak
\chapter{Programmi di calcolo}

Questo capitolo presenter\`a la catena di calcolo utilizzata per implementare
la trattazione \textit{ab initio} che verr\`a esposta nei suoi risultati nel
capitolo \ref{cap:applicazioni}. La catena \`e costituita dai seguenti
programmi
\begin{itemize}
\item \texttt{dalton 1.2}
\item \texttt{gaussian 98}
\item \texttt{ijkldali}
\item \texttt{cipselx}
\item \texttt{cippiS2s}
\item \texttt{koopro4E}
\item \texttt{dypc2}
\end{itemize}

Per le piattaforme di lavoro, si \`e fatto uso di Linux RedHat 7.2 su hardware
sia Intel che AMD. La configurazione Intel comprende un dual-processor 
Intel Pentium III a 1 GHz con 1 Gb di RAM ed un totale di 120 Gb di Hard-Disk
IDE/SCSI. La configurazione AMD comprende invece due macchine, 
un single-processor AMD
Athlon XP 1700+ con 1.5 Gb di RAM ed un totale di 90 Gb di Hard-Disk IDE
e un Athlon XP 1800+ con 1.5 Gb di RAM e 100 Gb totali di Hard-Disk IDE.


\section{Descrizione dei programmi}

\subsection{\texttt{dalton 1.2}}

Il programma \texttt{dalton 1.2} (Ref. \cite{dalton-site}) \`e stato utilizzato per la
maggior parte delle operazioni effettuate nella preparazione di questa tesi,
relativamente alla ottimizzazione geometrica e calcoli CASSCF. L'input \`e
caratterizzato da 2 file, con estensione .mol e .dal: il file .mol
contiene la geometria molecolare, in formato cartesiano xyz o con z-matrix,
e la base da utilizzare, per esempio
\begin{verbatim}
BASIS
cc-pVTZ
Formaldeide
cc-pVTZ
    3    2  X  Y
        8.    1
O      0.0000000000        0.0000000000        0.0264937595
        6.    1
C      0.0000000000        0.0000000000        2.3229964499
        1.    1
H      1.7539796658        0.0000000000        3.3947621243
\end{verbatim}

Il file .dal contiene invece le operazioni da eseguire, per esempio
\begin{verbatim}
**GENERAL
.OPTIMIZE
*OPTIMIZE
**DALTON INPUT
.RUN WAVE FUNCTION
**WAVE FUNCTIONS
.HF
.MP2
.MCSCF
*HF INPUT
.HF OCC
 5 2 1 0
*CONFIGURATION INPUT
.SYMMETRY
 1
.SPIN MUL
 1
.INACTIVE
 4 1 0 0
.ELECTRONS
 6
.CAS SPACE
 2 1 2 0
*END OF
\end{verbatim}
L'esempio qui riportato permette l'esecuzione di una ottimizzazione di geometria
su un calcolo CASSCF, usando come guess gli orbitali naturali di una trattazione HF/MP2.
L'input necessita, in alcuni casi, di accurate regole di posizionamento e
formattazione.

L'output di \texttt{dalton}, con estensione .out, contiene, tra le tante informazioni
disponibili, gli orbitali molecolari, la matrice densit\`a, l'energia dello
stato selezionato, la geometria ottimizzata. Oltre a questo file, in formato
testuale, vengono creati altri file operativi, solitamente in formato
binario. Tali file contengono varie informazioni che saranno successivamente utilizzate
dalla catena NEV-PT.

\subsection{\texttt{gaussian 98}}

Il programma \texttt{gaussian 98} (Ref. \cite{gaussian-site}) \`e un programma commerciale
che consente l'esecuzione di vari tipi di calcoli quantomeccanici. \`E stato utilizzato
nella costruzione della superficie di potenziale di etilene ed esatriene, in quanto pi\`u
robusto nell'imposizione dei vincoli e pi\`u facilmente automatizzabile
attraverso script di shell opportuni. Accetta in input un singolo file, con
estensione .g98, in cui \`e definita sia la z-matrix (o le coordinate
cartesiane) sia la procedura da eseguire. Lo stampato a seguire mostra un
esempio di input file per attuare una ottimizzazione geometrica vincolata
sulla molecola di etilene.
\begin{verbatim}
%Chk=c000-h000.chk
#p CASSCF(2,2) 6-31G* gfinput IOP(6/7=3) Guess=(read)
# pop=reg Opt(ModRedundant) NoSymm

etilene

   0   1
 H
 C,1,R2
 C,2,R3,1,A3
 H,3,R4,2,A4,1,D4,0
 H,3,R5,2,A5,1,D5,0
 H,2,R6,3,A6,5,D6,0
      Variables:

 R2=1.07570091
 R3=1.33882098
 R4=1.07570101
 R5=1.07579366
 R6=1.07579366
 A3=121.79543057
 A4=121.79551679
 A5=121.58188511
 A6=121.58189503
 D4=0.0
 D5=180.0
 D6=0.0

1 2 3 4 F
6 2 3 5 F

\end{verbatim}
L'output del programma \texttt{gaussian} pu\`o inoltre essere interpretato dal
programma grafico \texttt{molden} (cfr. \cite{molden}), che permette di visualizzare il percorso
di ottimizzazione geometrica e gli orbitali. \texttt{Molden} \`e stato utilizzato
per le rappresentazioni grafiche degli orbitali visibili in questo lavoro di tesi.

\subsection{\texttt{ijkldali}}

Il compito del programma \texttt{ijkldali} \`e effettuare una trasformazione
degli integrali mono e bielettronici forniti dal programma \texttt{dalton},
che li fornisce in base atomica,
restituendo gli integrali bielettronici su base molecolare.
Il programma fa uso dei file di lavoro di \texttt{dalton}, \texttt{AOONEINT},
\texttt{AOTWOINT} e \texttt{SIRIFC}. La linea di comando da eseguire viene
passata via standard input
\begin{verbatim}
&LEGGI DIR='/scr/dir/',
MCORE=450,ILOW=0,0,0,0,ZNEVPT=T,IOCC=10,9,2,1,
MOLAB='CASTOCIP',DALTOCIP=T,
&END
\end{verbatim}

L'opzione \texttt{DALTOCIP} \`e stata impostata a vero (\texttt{T}) quando si
richiedevano informazioni sullo spazio CI e sui coefficienti CI.

Il programma \texttt{ijkldali} necessita delle opzioni 
\texttt{.INTERFACE} e \texttt{.DETERMINANTS} nell'input file del \texttt{dalton}.

\subsection{\texttt{cipselx}}
\texttt{cipselx} viene utilizzato, in questa catena, per attuare un opportuno
ordinamento dei determinanti che definiscono lo spazio CI, riordinamento necessario
quando si utilizzino programmi che fanno uso delle configurazioni (Configuration State
Function, CSF) come ad esempio \texttt{cippiS2s}.
Anche in questo caso, la procedura da eseguire viene passata via standard
input
\begin{verbatim}
 &FILES FILE04='/scr/dir/file04',
        FILE03='/scr/dir/file03',
        FILE44='/scr/dir/FILE04',
        FILE25='/scr/dir/FILE25',   &END
 &ICINP ISZ=0,TEST=0.0000,ZAUTO=T,ZOLD=T,ZBIN=T,   &END
\end{verbatim}

\subsection{\texttt{cippiS2s}}

Nei calcoli qui riportati, viene utilizzato per diagonalizzare l'hamiltoniano
nello spazio CI fornito in uno dei file
operativi (\texttt{file04}) ottenuto da \texttt{dalton} attraverso successiva
elaborazione con \texttt{ijkldali}.
Tale programma costituisce l'ultima versione della procedura nota come
\texttt{CIPSI} (Configuration Interaction by Perturbation with
multiconfigurational zeroth-order wavefunction Selected by Iterative
process), sviluppata originariamente a Parigi e Tolosa
e perfezionata in seguito a Pisa e Ferrara.

\subsection{\texttt{koopro4E}}

Il programma si occupa del calcolo delle matrici di Koopmans e delle matrici
densit\`a ad una, due, tre e quattro particelle.

\subsection{\texttt{dypc2}}

Effettua il calcolo perturbativo NEV-PT vero e proprio. Restituisce in output
le energie CASSCF corrette al secondo ordine perturbativo con le partizioni
partially e strongly contracted, sia come contributo globale di
correzione all'energia, sia come contributi parziali per ogni classe perturbativa
derivante dagli operatori $\hat{\mathcal{V}}$, uniti alla norma della correzione
perturbativa al primo ordine alla funzione d'onda. 
Necessita in input delle matrici prodotte dal programma \texttt{koopro4E}.

