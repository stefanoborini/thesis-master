\pagebreak
\thispagestyle{empty}
{ \Large \textbf{Conclusioni} } 
\vspace{6mm} \\
In questo lavoro di tesi si sono eseguiti studi applicativi su sistemi di
varia natura, sfruttando la teoria perturbativa multireference NEV-PT. Gli
scopi prefissi sono stati raggiunti, ed in particolare si \`e dimostrata

\begin{itemize}
\item l'efficienza numerica della NEV-PT, che ottiene buoni risultati
mantenendo una buona scalabilit\`a, sia in termini di tempi macchina che in
termini di risorse hardware, questo grazie alla necessit\`a di effettuare
operazioni numeriche implicanti esclusivamente lo spazio attivo, di
dimensionalit\`a normalmente ridotta rispetto a quella della base orbitalica
totale.
\item la solidit\`a dell'algoritmo. In nessuno dei casi sottoposti ad
analisi si sono incontrate difficolt\`a computazionali come divergenze o
instabilit\`a.
\item l'applicabilit\`a dell'algoritmo perturbativo sia allo stato
fondamentale che agli stati eccitati. La descrizione di tali stati comporta
errori simili, in modo tale da rendere accettabile il risultato finale per
l'energia di transizione.
\item l'utilizzo di un approccio strongly contracted fornisce valori di poco
differenti da un'analoga trattazione partially contracted, con un costo
computazionale alquanto minore.
\item la risoluzione di molti problemi di tipo formale, 
quali la separabilit\`a stretta (size consistence) e l'assenza di stati
intrusi.
\end{itemize}

